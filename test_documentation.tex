\documentclass{article}

\usepackage{pdfpages}
\usepackage{fullpage}

\title{Documentation for the tests of KaMIS-ml}
\author{Joseph Holten}

\begin{document}

\maketitle

\section{Problem of interest}
The algorithms discussed all try to solve the maximum weighted independent set problem.


\section{Algorithms Tested}
The algorithms tested are:

\begin{itemize}
    \item \textit{iterative\_ml} A Reduce-and-Peel algorithm, using a machine learning heuristic for the peeling step.
    \item \textit{greedy\_reduction} A Reduce-and-Peel algorithm, using an engineered heuristic for the peeling.
    \item \textit{simple\_ml} A greedy algorithm, prioritising nodes based on a machine learning step.
    \item \textit{simple\_greedy} A greedy algorithm, prioritising based on the engineered heuristic.
\end{itemize}

\noindent The features used in the machine learning heuristic are as follows:

\begin{itemize}
    \item 
\end{itemize}

The machine learning was done with a random-forest approach, and a neural network approach.

\section{Random-Forest based Machine Learning}

\subsection{Initial Tests}
At first the machine learning was done using random forests. 
We tested whether machine learning heuristics could at all compete with the engineered heuristic in the peeling step.
For this, we compared iterative\_ml and greedy\_reduction in solution quality, the ratio of the solution produced by the algorithm and the optimal solution calculated by the Branch-and-Bound solver from KaMIS.
The set of graphs to be tested, consisted of unweighted graphs, assigned random weights. The method of weight generation was also varied, to observe changes in the results.

\includepdf[pages=-]{misc/charting/greedy_tests.pdf}
\includepdf[pages=-]{misc/charting/greedy_ml_tests.pdf}

\end{document}
